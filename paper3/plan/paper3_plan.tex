
\documentclass[11pt]{article}
\usepackage{amsmath, amssymb, graphicx, hyperref, booktabs}
\usepackage[margin=1in]{geometry}

\title{Paper III Research Plan:\\
Transient Dimensional Inflation via Timescale Separation in Energy-Coupled Chemical Oscillators}
\author{}
\date{\today}

\begin{document}

\maketitle

\section*{Abstract}

Papers I--II established that (i) network growth does not inflate attractor dimensionality and (ii) topology-aware growth preserves oscillatory viability without increasing dynamical dimensionality. Together they reveal a pattern of \emph{dynamical rigidity}: autocatalytic oscillators resist dimensional inflation under both stoichiometric growth (Paper I) and topological selection (Paper II). A key structural insight emerged: \emph{dimensionality of state space $\ne$ dimensionality of attractor}---adding variables or reactions does not guarantee higher-dimensional dynamics. Pilot experiments confirmed this: fast-relaxing energy drive does not break rigidity (Pilots 1--3); slow energy can (Pilot 3b, $D_2 = 1.848$) but only in a fragile, narrow parameter window.

Pilot 4 demonstrated that two oscillatory modules sharing a slow energy variable produce robust $D_2 > 1$, and Pilot 5b validated this within the \texttt{ReactionSimulator} mass-action framework ($D_2 = 1.660$). However, the independent-reservoir control (Sweep A2) revealed that at very slow energy relaxation ($\gamma = 0.001$), each oscillator-energy subsystem inflates \emph{independently}---sharing is incidental. Sliding-window diagnostics further revealed that the high-$D_2$ phase is \textbf{transient}: the system explores a complex attractor for hundreds of oscillator periods before phase-locking restores rigidity. The primary mechanism is \textbf{timescale separation enabling long-lived transient escape from dynamical rigidity}. Sharing prolongs the transient and enables richer complexity at moderate $\gamma$. This document formalizes the experimental plan for Paper III.

\subsection*{Series continuity}

The three papers form a progression through the rigidity landscape:
\begin{enumerate}
\item \textbf{Growth rigidity} (Paper I): Adding reactions dilutes $\eta$; $D_2$ stays $\approx 1$.
\item \textbf{Topological rigidity} (Paper II): Feedback-aligned growth preserves oscillation but cannot inflate $D_2$.
\item \textbf{Energetic rigidity} (Pilot 3): Energy injection into a single oscillator does not inflate $D_2$ when $E$ relaxation is fast (dynamically slaved).
\item \textbf{Transient escape from rigidity} (Paper III): A sufficiently slow energy variable ($1/\gamma \gg$ oscillator period) enables long-lived transient high-dimensional dynamics before phase-locking restores rigidity. Sharing the slow variable prolongs the exploratory phase and enables richer transient complexity at moderate $\gamma$.
\end{enumerate}

To preserve this continuity, the coupled oscillator model is implemented as an explicit mass-action reaction network using the same \texttt{GeneratedNetwork} / \texttt{ReactionSimulator} / \texttt{ActivationTracker} stack as Papers I--II. This enables the activation ratio $\eta = D_2 / r_s$ to serve as the common measure across all three papers, with growth applied through the same \texttt{generate\_progressive} / \texttt{generate\_progressive\_aligned} operators.

\subsection*{Operational definition}

\textbf{Dimensional inflation} is defined operationally as \emph{time-local} $D_2 > 1.2$ in at least one sliding window, across $\ge 3$ independent seeds at a given parameter set. This threshold is justified statistically, not heuristically: across all pilot runs classified as limit cycles (Pilots 1--5, $N > 200$), the estimator returns $D_2 \approx 1.00 \pm 0.03$ (mean $\pm$ SD). The threshold $D_2 = 1.2$ is therefore $>6\sigma$ above the limit-cycle baseline, making false positives negligible. Dynamically, $D_2 > 1.2$ in a given window indicates transient dynamics on a higher-dimensional manifold (torus or strange attractor) rather than a limit cycle.

\textbf{Transient duration metrics} (added post-diagnostics):
\begin{itemize}
\item \textbf{Locking time} $T_{\text{lock}}$: the first time $t$ after which the system remains phase-locked. Operationally: the earliest $t$ such that sliding-window median $D_2 < 1.1$ and $|r(X_1, X_2)| > 0.95$ for all subsequent windows. Measures how long the system explores before settling.
\item \textbf{Exploratory duration} $\tau_{>1.2}$: total time (or number of windows) with $D_2 > 1.2$. The primary ecology metric---quantifies how long chemistry stays complex before locking.
\end{itemize}

\textbf{Phase 0.5 calibration task}: As part of ridge mapping, compute the sample mean and SD of $D_2$ across all Phase 0.5 runs classified as phase-locked limit cycles. Report the exact baseline distribution in the paper to anchor the 1.2 threshold empirically.

\section{Model Specification}

\subsection*{Architecture commitment}

All models in Paper III are implemented as explicit mass-action reaction networks compatible with the \texttt{ReactionSimulator} framework. No parallel ODE universe.

\textbf{Key design decision}: The pilot model (Pilot 4) used Michaelis--Menten gating $f(E) = E/(K+E)$, which is not representable as standard mass-action. Two approaches were tested:
\begin{itemize}
\item \textbf{Pilot 5 (FAILED)}: Pure mass-action E-consumption ($k \cdot E \cdot X_i^2 Y_i$, linear in $E$). Max $D_2 = 0.991$. Linear coupling creates a binary regime (phase-locked or anti-phase) without the intermediate quasi-periodic window.
\item \textbf{Pilot 5b (VALIDATED)}: Enzyme-complex representation using auxiliary gate species $G$ and complex $GE$. Max $D_2 = 1.660$. Reproduces Michaelis--Menten saturation entirely within mass-action kinetics.
\end{itemize}

\textbf{Lesson}: Michaelis--Menten saturation is not just convenient---it is dynamically essential. The saturation bounds effective coupling at high $E$, creating the narrow window for timescale separation that enables quasi-periodic dynamics. The enzyme-complex trick achieves this bounding within mass-action.

\subsection*{Enzyme-complex coupled model}

Gate species $G$ and enzyme-substrate complex $GE$ mediate energy access to the autocatalytic cores:

\begin{align*}
G + E &\rightleftharpoons GE \qquad \text{(fast equilibrium, } K_d = k_{\text{off}} / k_{\text{on}}\text{)} \\
2X_i + Y_i + GE &\to 3X_i + G + E_w \qquad \text{(gated autocatalysis, E consumed)}
\end{align*}

At quasi-steady-state for $GE$:
\[
[GE]_{\text{ss}} \approx G_{\text{total}} \cdot \frac{E}{K_d + E}
\]

Effective autocatalytic rate: $k_{\text{cat}} \cdot G_{\text{total}} \cdot \frac{E}{K_d + E} \cdot X_i^2 Y_i$. This \textbf{is} Michaelis--Menten. $E$ is irreversibly consumed ($\to E_w$), creating true resource coupling. $G$ is regenerated (catalytic). $G + GE = G_{\text{total}}$ is conserved.

\subsection*{Complete reaction list}

Per core $i \in \{1, 2\}$, using the Paper I--II Brusselator notation:

\begin{center}
\begin{tabular}{llll}
\toprule
Rxn & Reaction string & Rate law & Role \\
\midrule
R1$_i$ & \texttt{A -> Xi} & $A$ (chemostatted) & Food inflow \\
R2$_i$ & \texttt{B + Xi -> Yi + Di} & $B \cdot X_i$ & Conversion \\
R3$_i$ & \texttt{Xi + Xi + Yi -> Xi + Xi + Xi} & $X_i^2 Y_i$ & Base autocatalysis \\
R4$_i$ & \texttt{Xi -> Wi} & $X_i$ & Decay \\
\midrule
R5 & \texttt{Esrc -> Esrc + E} & $J$ (Esrc chemostatted at 1) & Energy inflow \\
R6 & \texttt{E -> Ew} & $\gamma \cdot E$ & Energy leak \\
\midrule
R7 & \texttt{G + E -> GE} & $k_{\text{on}} \cdot G \cdot E$ & Gate binding \\
R8 & \texttt{GE -> G + E} & $k_{\text{off}} \cdot GE$ & Gate unbinding \\
\midrule
R9$_i$ & \texttt{Xi+Xi+Yi+GE -> Xi+Xi+Xi+G+Ew} & $k_{\text{cat}} \cdot X_i^2 Y_i \cdot GE$ & Gated autocatalysis \\
\bottomrule
\end{tabular}
\end{center}

Total: 14 reactions (4 per core $\times$ 2 + 2 energy pool + 2 gate + 2 gated autocatalysis).

Species: $\{X_1, Y_1, D_1, W_1, X_2, Y_2, D_2, W_2, E, E_w, G, GE\}$ (12 total) plus chemostatted $\{A, B, E_{\text{src}}\}$.

Dynamic species tracked for $D_2$: $\{X_1, Y_1, X_2, Y_2, E\}$ (5 variables). $G$ and $GE$ are fast auxiliary variables excluded from $D_2$ projection. Waste species $D_i, W_i, E_w$ are monotonic accumulators, automatically excluded by \texttt{ActivationTracker.\_extract\_trajectory()}.

\subsection*{Effective ODE (at quasi-steady-state for GE)}

When $k_{\text{on}}, k_{\text{off}} \gg k_{\text{cat}} \cdot X_i^2 Y_i$ (fast gate equilibrium), the effective dynamics reduce to:
\begin{align*}
\dot X_i &= A - (B+1)X_i + X_i^2 Y_i + k_{\text{cat}} \cdot \frac{G_{\text{total}} \cdot E}{K_d + E} \cdot X_i^2 Y_i \\
\dot Y_i &= B X_i - X_i^2 Y_i - k_{\text{cat}} \cdot \frac{G_{\text{total}} \cdot E}{K_d + E} \cdot X_i^2 Y_i \\
\dot E &= J - \gamma E - k_{\text{cat}} \cdot \frac{G_{\text{total}} \cdot E}{K_d + E} \cdot (X_1^2 Y_1 + X_2^2 Y_2)
\end{align*}

This recovers Pilot 4's Michaelis--Menten model with the identification:
\[
k_{\text{extra}} \leftrightarrow k_{\text{cat}} \cdot G_{\text{total}}, \qquad K \leftrightarrow K_d = k_{\text{off}} / k_{\text{on}}, \qquad \alpha \leftrightarrow k_{\text{cat}} \cdot G_{\text{total}}
\]

\subsection*{Parameters}

\begin{center}
\begin{tabular}{lll}
\toprule
Parameter & Role & Pilot 5b validated value \\
\midrule
$A, B$ & Brusselator feed (chemostatted) & 1.0, 3.0 \\
$J$ & Energy inflow rate & 5.0 \\
$\gamma$ & Energy leak rate & 0.002 \\
$k_{\text{on}}$ & Gate binding rate & 10.0 \\
$k_{\text{off}}$ & Gate unbinding rate & 10.0 ($\Rightarrow K_d = 1.0$) \\
$k_{\text{cat}}$ & Catalytic rate (gated autocatalysis) & 0.3 \\
$G_{\text{total}}$ & Total gate concentration ($G + GE$) & 1.0 \\
\bottomrule
\end{tabular}
\end{center}

\subsection*{Stoichiometric rank}

$r_s$ = rank of the $12 \times 14$ stoichiometric matrix. Computed identically to Papers I--II via \texttt{StoichiometricAnalyzer.from\_reaction\_strings()}.

\subsection*{Recovery conditions}

At $k_{\text{cat}} = 0$ or $J = 0$: $E \to 0$, $GE \to 0$, R9$_i$ terms vanish, each core reduces to the standard Brusselator. All dimensional inflation is attributable to modular energy coupling via the enzyme-complex mechanism.

\section{Pilot 5b Findings and Interpretive Framework}

\subsection*{Validated result}

Pilot 5b achieved $D_2 = 1.660$ (median 1.558, 3/3 seeds complex) at $J = 5.0$, $\gamma = 0.002$, $k_{\text{cat}} = 0.3$, $K_d = 1.0$, $G_{\text{total}} = 1.0$. Run entirely within \texttt{ReactionSimulator}.

\subsection*{Critical observation: gate saturation}

At the validated parameter set, $E_{\text{mean}} \approx 445$, so $f(E) = E/(K_d + E) \approx 0.998$. The gate is \textbf{fully saturated}. This means:

\textbf{The dimensional inflation is not coming from nonlinear gating curvature.} The Michaelis--Menten function is effectively constant at $\approx 1$.

Instead, the inflation arises from:
\begin{itemize}
\item Extremely high energy availability ($J/\gamma = 2500$)
\item Very slow relaxation ($1/\gamma = 500$ time units)
\item Weak-to-moderate catalytic coupling ($k_{\text{cat}} = 0.3$)
\end{itemize}

The role of Michaelis--Menten saturation is \emph{structural}: it \textbf{bounds} the effective coupling so that it cannot grow without limit as $E$ increases. Without this bounding (Pilot 5, linear in $E$), coupling strength scales with $E$, synchronizing the oscillators at high $E$. With saturation, coupling plateaus at $k_{\text{cat}} \cdot G_{\text{total}}$ regardless of $E$, allowing the slow $E$ dynamics to modulate without overwhelming.

\textbf{Framing caution (anticipated reviewer attack)}: At $J = 5.0$, the gate is saturated and the coupling term is effectively constant. A reviewer may object: ``The system reduces to two weakly coupled oscillators---why is this surprising?'' The answer: the claim is \emph{not} that enzyme gating or modular sharing produces complexity. \textbf{Sweep A2 proved this}: each oscillator-energy subsystem inflates independently. The claim is that \textbf{timescale separation}---a slow energy variable coupled to a fast oscillator---is sufficient to break dynamical rigidity. The enzyme-complex mechanism is the means of embedding this in mass-action; the dynamically essential ingredient is a slow degree of freedom ($1/\gamma \gg$ oscillator period) with bounded coupling that prevents synchronization. Sharing this slow variable between modules is biologically natural and creates additional modulatory effects, but the fundamental mechanism is slowness, not sharing.

\subsection*{Two distinct inflation regimes}

The data reveals two qualitatively different regimes of $D_2 > 1$, unified by the requirement for \textbf{timescale separation}:

\begin{enumerate}
\item \textbf{Intrinsic slow--fast inflation} ($\gamma = 0.001$, Sweep A + A2): Each oscillator-energy subsystem ($X_i, Y_i, E_i$) independently produces $D_2 \approx 1.40$. The slow $E$ ($1/\gamma = 1000$ time units) creates genuine slow--fast dynamics within each 3D subsystem---likely quasi-periodic torus behavior from amplitude modulation. Sharing the reservoir is incidental; independent reservoirs produce the same $D_2$. This regime is robust across all $J$ values and most $k_{\text{cat}}$ values. \textbf{Confirmed by Sweep A2.}

\item \textbf{Shared-reservoir desynchronization inflation} (moderate $\gamma \approx 0.002$--$0.003$, Sweeps A + A2b): Energy relaxation is slower than the oscillator but not slow enough for independent inflation ($D_2 \approx 0.99$ per independent subsystem). Oscillators sharing this E pool desynchronize ($r \approx 0$) and $D_2$ reaches its highest values (up to 1.67). The shared reservoir IS causally essential: independent reservoirs produce only $D_2 \approx 0.994$ ($\Delta = -0.57$). \textbf{Confirmed by Sweep A2b.}
\end{enumerate}

Both regimes require a slow energy variable. The unifying mechanism is \emph{timescale separation}. In Regime 1, slowness alone is sufficient. In Regime 2, slowness is necessary but not sufficient---the shared reservoir creates inter-oscillator competition that provides an additional slow timescale (desynchronization/beating) enabling the highest-dimensional attractors.

\subsection*{Phase 0.5: Ridge mapping (pre-registered micro-tests)}

Before the full Phase I sweep, we must establish that the $D_2 > 1.2$ region is not razor-thin. Pilot 5b found only 1 of 39 parameter sets with $D_2 > 1.2$, with near-misses at $D_2 = 0.992$ (J=3, J=5 with $k_{\text{cat}}=0.5$).

\textbf{Ridge mapping protocol}: Probe the neighborhood of the validated point ($J=5$, $\gamma=0.002$, $k_{\text{cat}}=0.3$):

\begin{center}
\begin{tabular}{lll}
\toprule
Parameter & Values & Rationale \\
\midrule
$J$ & $\{4.0, 4.5, 5.0, 5.5, 6.0\}$ & Bracket the J=5 sweet spot \\
$\gamma$ & $\{0.002, 0.003, 0.004\}$ & Test sensitivity to E timescale \\
$k_{\text{cat}}$ & $\{0.20, 0.25, 0.30, 0.35\}$ & Bracket the $k_{\text{cat}} = 0.3$ sweet spot \\
\bottomrule
\end{tabular}
\end{center}

Grid: $5 \times 3 \times 4 = 60$ parameter sets $\times$ 3 seeds = 180 runs ($\sim$6 hours).

\textbf{Definition}: The \textbf{ridge holds} if $\ge 3$ parameter sets (not just the validated point) achieve $D_2 > 1.2$ across $\ge 3$ seeds each, with the $D_2$ convergence criterion met ($|D_2(2T) - D_2(T)| < 0.05$), and these sets span a connected neighborhood in $(J, \gamma, k_{\text{cat}})$ space---where ``connected'' means grid-adjacent (Manhattan distance 1 on the discrete $5 \times 3 \times 4$ lattice), i.e., the inflation region is not isolated points.

\textbf{Success criteria}:
\begin{itemize}
\item If ridge holds with $\ge 5$ parameter sets $\to$ robust; proceed to Phase I with full Lyapunov analysis.
\item If $D_2 > 1.2$ confined to 1--2 sets $\to$ ridge is narrow; expand $J$ range and add $K_d$ variation before Phase I.
\item If no $D_2 > 1.2$ outside $(J=5, \gamma=0.002, k_{\text{cat}}=0.3)$ $\to$ re-examine model; consider longer integration or finer seeds.
\end{itemize}

\subsection*{Dynamical classification diagnostics}

Before writing claims about the nature of the $D_2 > 1$ attractor, verify at the best Phase 0.5 parameter set(s). The primary concern is distinguishing genuine higher-dimensional dynamics from transient beating or slow amplitude modulation.

\begin{enumerate}
\item \textbf{Power spectrum}: Compute FFT of $X_1(t)$ and $X_2(t)$ post-transient. Look for two incommensurate peaks (torus) vs.\ broadband spectrum (chaos) vs.\ single peak with harmonics (limit cycle with slow modulation).
\item \textbf{Phase portrait projection}: Plot $X_1$ vs $X_2$ in the post-transient window. A filled annulus suggests torus; a structured strange attractor suggests chaos; a closed curve suggests phase-locked limit cycle.
\item \textbf{$D_2$ convergence under integration time doubling}: Re-run the best parameter set at $t_{\text{span}} = [0, T]$ for $T \in \{10{,}000,\; 20{,}000,\; 40{,}000\}$. Report $D_2(T)$ at each duration. \textbf{Pass criterion}: $|D_2(2T) - D_2(T)| < 0.05$ for both steps. If $D_2$ drifts monotonically downward, the regime is transient; if upward, the attractor may be weakly chaotic with slow exploration. This test directly addresses the reviewer objection that ``$D_2 > 1.2$ may reflect finite-time effects.''
\item \textbf{Lyapunov exponent sign}: Estimate the maximal Lyapunov exponent $\lambda_1$ via the Rosenstein algorithm applied to the 5D projected trajectory ($X_1, Y_1, X_2, Y_2, E$), using $\ge 10{,}000$ post-transient sample points (embedding dimension $m = 5$, delay chosen by first minimum of mutual information). If the ODE Jacobian is analytically available, cross-validate with the variational equation method. \textbf{Acceptance language}: $\lambda_1 > 0$ (significantly positive, bootstrapped 95\% CI excludes zero) $\to$ chaos; $\lambda_1 \approx 0$ (CI includes zero) $\to$ consistent with torus/quasi-periodic; $\lambda_1 < 0$ $\to$ limit cycle. Report the sign, point estimate, and CI. This is \textbf{required} if the ridge holds; optional only if the ridge collapses. The sign of $\lambda_1$ is the minimal evidence needed to distinguish true quasi-periodicity from weakly chaotic amplitude modulation.
\item \textbf{Sliding-window $D_2$ profile} (required): Run a single long integration ($T = 20{,}000$) and compute $D_2$ on non-overlapping 2500-unit windows. Report: max $D_2$, median $D_2$, number of windows above threshold, $T_{\text{lock}}$, and $\tau_{>1.2}$. This is the primary tool for characterizing the transient high-dimensional phase. \textbf{Post-diagnostics result}: At both exemplar points, $D_2 > 1.2$ is confined to $t < 10{,}000$ (Regime 1) or $t < 7{,}500$ (Regime 2), confirming the complex phase is transient.
\end{enumerate}

These diagnostics do \emph{not} gate Phase I (which proceeds regardless), but they inform the paper's dynamical claims. The sliding-window profile is now the definitive diagnostic: it shows that $D_2 > 1.2$ reflects a \emph{long-lived transient} (hundreds of oscillator periods of genuine chaos, confirmed by positive $\lambda_1$) rather than an asymptotic attractor. The paper claims ``metastable high-dimensional exploratory phase'' rather than ``permanent dimensional escape.''

\subsection*{Sweep A2: Independent-reservoir causal control at $\gamma = 0.001$}

Sweep A revealed that the dominant inflation regime is the $\gamma = 0.001$ band (15 of 22 complex parameter sets), where oscillators are \emph{synchronized} ($r \approx 0.9$+) and the shared $E$ is extremely slow ($1/\gamma = 1000$ time units). This shifts the mechanistic interpretation: inflation is not primarily from competition but from \textbf{slow-manifold dimensional expansion}---the shared reservoir acts as a genuine slow dynamical variable.

This makes the independent-reservoir control (Control 4 from Phase I-B) \textbf{urgent}. If splitting $E$ into $E_1, E_2$ kills $D_2 > 1.2$ at $\gamma = 0.001$, the shared reservoir is causally responsible. If $D_2$ persists, the inflation comes from adding a slow variable to each oscillator independently, and the thesis must be revised.

\textbf{Sweep A2 protocol}: Run the independent-reservoir variant at the top Sweep A parameter sets from the $\gamma = 0.001$ band:

\begin{center}
\begin{tabular}{lll}
\toprule
Parameter & Values & Rationale \\
\midrule
$J$ & $\{3, 5, 7, 10\}$ & Span the J range \\
$\gamma$ & $\{0.001\}$ & The dominant inflation regime \\
$k_{\text{cat}}$ & $\{0.1, 0.3, 0.5\}$ & Low, mid, high coupling \\
\bottomrule
\end{tabular}
\end{center}

Grid: $4 \times 1 \times 3 = 12$ parameter sets $\times$ 5 seeds = 60 runs.

Model modification: replace shared $E$, $G$, $GE$ with per-core $E_i$, $G_i$, $GE_i$ ($i \in \{1,2\}$). Each core has its own energy source, leak, gate binding/unbinding, and gated autocatalysis. The two cores are completely decoupled. Track the 6D projection $(X_1, Y_1, E_1, X_2, Y_2, E_2)$ for $D_2$.

\textbf{Decision criteria}:
\begin{itemize}
\item If $D_2 > 1.2$ disappears across all 12 sets $\to$ shared reservoir is \textbf{causally essential}. Proceed with the ``slow shared resource coupling'' narrative.
\item If $D_2 > 1.2$ persists at comparable levels $\to$ inflation is intrinsic to each 3D oscillator-energy subsystem. The shared-reservoir claim must be weakened to ``slow energy coupling inflates dimensionality regardless of sharing.''
\item If $D_2$ is intermediate (e.g., $D_2 \approx 1.0$--$1.2$ per subsystem, $\sim$2.0 combined but decomposable) $\to$ the shared reservoir provides non-trivial coupling that lifts $D_2$ above the sum of parts.
\end{itemize}

\textbf{This test runs before Sweeps B/C.} It determines whether the central thesis survives.

\subsubsection*{Sweep A2 result (COMPLETE)}

\textbf{D$_2 > 1.2$ persists with independent reservoirs.} 11 of 12 parameter sets show $D_2 > 1.2$ with completely decoupled energy pools. Key diagnostics:
\begin{itemize}
\item $D_{2,\text{6D}} \approx D_{2,\text{core1}} \approx D_{2,\text{core2}} \approx 1.40$--$1.44$: each 3D subsystem inflates independently.
\item $D_{2,\text{4D}}$ (oscillators only, no E) $\approx 0.99$: the slow $E$ is what inflates, not the oscillator coupling.
\item Sharing sometimes \emph{enhances} (J=7, $k_{\text{cat}}=0.5$: shared $D_2 = 1.542$ vs independent $1.400$) and sometimes \emph{suppresses} (J=3, $k_{\text{cat}}=0.3$: shared $D_2 = 0.992$ vs independent $1.410$).
\end{itemize}

\textbf{Interpretation}: At $\gamma = 0.001$ ($1/\gamma = 1000$ time units), the slow energy variable creates genuine timescale separation \emph{within each subsystem}---this is Pilot 3b's slow--fast bursting mechanism, not a coupling phenomenon. The shared reservoir is incidental at these parameters.

\textbf{Thesis impact}: The $\gamma = 0.001$ regime does \textbf{not} support the ``shared resource coupling'' claim. However, the moderate-$\gamma$ desynchronized regime (Regime 2: $J=5$, $\gamma=0.002$, $k_{\text{cat}}=0.3$, $D_2 = 1.558$, $r \approx 0$) was not tested and may still require sharing.

\subsection*{Sweep A2b: Independent-reservoir control at moderate $\gamma$ (COMPLETE)}

The decisive test for shared-coupling causality at moderate $\gamma$. Grid: $J \in \{4, 5\}$, $\gamma \in \{0.002, 0.003\}$, $k_{\text{cat}} \in \{0.2, 0.3\}$ = 8 sets $\times$ 5 seeds = 40 runs. Completed in 21 min.

\subsubsection*{Sweep A2b result}

\textbf{D$_2 > 1.2$ DISAPPEARS with independent reservoirs at moderate $\gamma$.} All 8 parameter sets show $D_2 \approx 0.994$---standard limit cycle. The two key shared-reservoir inflation points collapse completely:
\begin{itemize}
\item $J = 4$, $\gamma = 0.003$, $k_{\text{cat}} = 0.2$: shared $D_2 = 1.566 \to$ independent $D_2 = 0.993$ ($\Delta = -0.573$)
\item $J = 5$, $\gamma = 0.002$, $k_{\text{cat}} = 0.3$: shared $D_2 = 1.558 \to$ independent $D_2 = 0.994$ ($\Delta = -0.565$)
\end{itemize}

\textbf{Interpretation}: At moderate $\gamma$ ($1/\gamma = 333$--$500$), E relaxation is slow but \textbf{not slow enough} for each subsystem to independently inflate. When two oscillators \textbf{share} this E pool, inter-oscillator competition creates an additional slow timescale---desynchronization/beating---that inflates $D_2$ to $1.56$+. This is genuine coupling-mediated complexity.

\textbf{Combined A2 + A2b verdict}: Two distinct mechanisms confirmed with clean causal separation:
\begin{enumerate}
\item \textbf{Intrinsic timescale-separation} ($\gamma = 0.001$): Each 3D subsystem inflates independently ($D_2 \approx 1.4$). Sharing incidental.
\item \textbf{Shared-reservoir desynchronization} ($\gamma = 0.002$--$0.003$): Sharing IS essential ($D_2$ up to 1.67). Independent subsystems produce $D_2 \approx 1.0$ only.
\end{enumerate}

\section{Phase I: Fine Grid Sweep}

\subsection*{Objective}
Map the $D_2 > 1.2$ boundary precisely in parameter space and confirm that the dimensional inflation window is not seed-dependent.

\textbf{All Phase I runs use \texttt{ReactionSimulator} with the enzyme-complex network.} This ensures the same code path as the growth experiments.

\subsection*{Sweep design (to be refined by Phase 0.5 results)}

\paragraph{Sweep A --- Primary parameter scan.}
Fix $K_d = 1.0$, $G_{\text{total}} = 1.0$, $k_{\text{on}} = k_{\text{off}} = 10.0$. Vary:
\begin{center}
\begin{tabular}{lll}
\toprule
Parameter & Values & Rationale \\
\midrule
$J$ & $\{3, 4, 5, 7, 10\}$ & Pilot 5b: complex at J=5, near-miss at J=3 \\
$\gamma$ & $\{0.001, 0.002, 0.003, 0.005\}$ & Only $\gamma \le 0.005$ in pilots \\
$k_{\text{cat}}$ & $\{0.1, 0.2, 0.3, 0.4, 0.5\}$ & Bracket the $k_{\text{cat}} = 0.3$ sweet spot \\
\bottomrule
\end{tabular}
\end{center}
Grid size: $5 \times 4 \times 5 = 100$ parameter sets.

\paragraph{Sweep B --- Gate properties.}
Fix $J = 5.0$, $\gamma = 0.002$, $k_{\text{cat}} = 0.3$. Vary:
\begin{center}
\begin{tabular}{lll}
\toprule
Parameter & Values & Rationale \\
\midrule
$K_d$ & $\{0.3, 1.0, 3.0, 10.0\}$ & Half-saturation threshold \\
$G_{\text{total}}$ & $\{0.5, 1.0, 2.0\}$ & Effective max coupling rate \\
\bottomrule
\end{tabular}
\end{center}
Grid size: $4 \times 3 = 12$ parameter sets.

\paragraph{Sweep C --- Gate equilibrium speed.}
Fix best primary params, vary $k_{\text{on}} = k_{\text{off}} \in \{1, 10, 100\}$ (holding $K_d = 1.0$ constant). Tests whether the QSS approximation holds across gate timescales. Grid: 3 parameter sets.

\subsection*{Seeds and runtime}

\begin{itemize}
\item $N_{\text{seeds}} = 5$ per parameter set (up from 3 in pilot; sufficient for median stability).
\item Initial conditions: $(X_1, Y_1) = (1.0, 1.0)$, $(X_2, Y_2)$ perturbed by $\mathcal{U}(-0.05, 0.05)$ per seed, $E(0) = J / \gamma$, $GE(0) = G_{\text{total}} \cdot E_0 / (K_d + E_0)$, $G(0) = G_{\text{total}} - GE(0)$.
\item Total runs: $(100 + 12 + 3) \times 5 = 575$.
\item Estimated runtime: $\sim$20 s/run $\times$ 575 = $\sim$3.2 hours.
\end{itemize}

\subsection*{Integration parameters}

\begin{itemize}
\item Integrator: \texttt{ReactionSimulator.simulate()} (uses \texttt{solve\_ivp} internally).
\item $t_{\text{span}} = [0, 10{,}000]$ time units.
\item Transient discard: first 50\%.
\item Sampling: 20{,}000 equally spaced points.
\end{itemize}

\subsection*{Evaluation metrics}

Per parameter set (across seeds):
\begin{itemize}
\item Regime classification: fixed\_point / phase\_locked / complex (same criteria as Pilot 4).
\item Median and max $D_2$ across seeds.
\item Fraction of seeds with $D_2 > 1.2$.
\item Phase correlation $r(X_1, X_2)$.
\item Energy statistics: $E_{\text{mean}}$, $E_{\text{range}}$, $GE_{\text{mean}}$, gate saturation $\langle GE \rangle / G_{\text{total}}$.
\item $r_s$ and $\eta = D_2 / r_s$ at baseline (no growth).
\item Sliding-window $D_2$ profile: max $D_2$, $\tau_{>1.2}$ (windows above threshold), $T_{\text{lock}}$ (locking time).
\end{itemize}

\subsection*{Deliverable}
Heatmaps of median $D_2$ in $(\gamma, k_{\text{cat}})$ at each $J$ level. Annotated boundary of the $D_2 > 1.2$ region. Gate saturation map to confirm whether the abundance-regime interpretation holds broadly. Heatmap of $\tau_{>1.2}$ showing which parameter sets produce the longest exploratory phases.

\section{Phase I-B: Control Experiments}

Four controls establish the causal role of timescale separation and characterize the modulatory effect of shared reservoirs.

\subsection*{Control 1: Single Brusselator + E + G/GE}

Model: Single-core version of the enzyme-complex network (R1--R4 for one core only + R5--R9 for one core). Same $J, \gamma, k_{\text{cat}}, K_d, G_{\text{total}}$ as best Phase I parameter set.

Prediction: $D_2 > 1.2$ at $\gamma = 0.001$ (consistent with Sweep A2: single subsystem inflates via timescale separation). $D_2 \approx 1$ at moderate $\gamma$ unless parameters are extreme (cf.\ Pilot 3b's narrow window). This control now \emph{confirms} the intrinsic slow-energy mechanism rather than testing modularity.

\subsection*{Control 2: Two uncoupled Brusselators (no energy pool)}

Model: Set $k_{\text{cat}} = 0$ (remove R9$_i$), retain R5--R8 (energy pool and gate evolve but do not influence cores). Two independent Brusselators.

Prediction: $D_2 \approx 1$ (each core is a limit cycle; no interaction $\to$ no new attractor structure). Confirms that coupling through $E$ is necessary.

\subsection*{Control 3: Two Brusselators with direct diffusive coupling (no slow E)}

Model: Replace energy pool with direct coupling: add reactions $X_1 \to X_2$ and $X_2 \to X_1$ at rate $\epsilon$. No $E$, $G$, $GE$ variables. This is mass-action representable.

Prediction: Phase locking at most coupling strengths ($D_2 \approx 1$), possible anti-phase at intermediate $\epsilon$. Tests whether slow energy specifically is needed vs.\ any coupling mechanism.

\subsection*{Control 4: Two Brusselators with independent energy pools $E_1$, $E_2$}

Model: Replace the single shared $E$ with two independent energy pools $E_1, E_2$, each with its own source, leak, and gate:
\begin{itemize}
\item $E_{\text{src}} \to E_{\text{src}} + E_i$ (rate $J$), $E_i \to E_w$ (rate $\gamma \cdot E_i$), for $i \in \{1, 2\}$.
\item $G_i + E_i \to GE_i$, $GE_i \to G_i + E_i$, and R9$_i$ uses $GE_i$ instead of shared $GE$.
\end{itemize}
Same $J, \gamma, k_{\text{cat}}, K_d, G_{\text{total}}$ per pool as the best Phase I parameter set. The two cores are now \textbf{completely decoupled}: each core + its own energy pool is a 3D subsystem.

\textbf{This is the critical shared-reservoir necessity test.}
\begin{itemize}
\item If $D_2 > 1.2$ \textbf{disappears} $\to$ the \emph{shared} reservoir is essential for dimensional inflation. The claim ``shared slow resource coupling breaks rigidity'' is confirmed.
\item If $D_2 > 1.2$ \textbf{persists} $\to$ modularity alone is sufficient (each independent 3D oscillator already has $D_2 > 1$ intrinsically), and the shared-reservoir narrative must be revised. This would undermine the central thesis and require reformulation.
\end{itemize}

\textbf{Updated expectation (post-Sweep A2)}: At $\gamma = 0.001$, $D_2 > 1.2$ per subsystem (already confirmed by Sweep A2). At moderate $\gamma$, $D_2$ may drop below 1.2 per subsystem (pending Sweep A2b). The combined 6D trajectory will show $D_2$ approximately equal to each 3D subsystem's $D_2$ (trivially decomposable), which is qualitatively different from the quasi-periodic attractor structure seen with shared $E$ at moderate $\gamma$.

\subsection*{Control results (COMPLETE)}

All 4 controls passed their predictions. 75 standard runs + 2 sliding-window exemplars completed in 12.8 min.

\begin{itemize}
\item \textbf{Control 1} (Single core + E): Median $D_2 = 1.400$, 15/25 complex (all at $\gamma = 0.001$). Sliding-window: $T_{\text{lock}} = 7500$ (J=7), $T_{\text{lock}} = 5000$ (J=5). \textbf{Confirms intrinsic slow-energy mechanism.}
\item \textbf{Control 2} (Uncoupled, $k_{\text{cat}} = 0$): Median $D_2 = 0.990$, 0/25 complex. \textbf{Confirms coupling through E is necessary.}
\item \textbf{Control 3} (Diffusive, no E): Median $D_2 = 0.991$, 0/25 complex at all $\epsilon \in \{0.01, 0.05, 0.1, 0.5, 1.0\}$. \textbf{Confirms slow E is essential (not just any coupling).}
\item \textbf{Control 4} (Independent reservoirs, from Sweeps A2 + A2b): $D_2$ persists at $\gamma = 0.001$ (55/60 complex), disappears at moderate $\gamma$ (0/40 complex). \textbf{Confirms two distinct mechanisms.}
\end{itemize}

\subsection*{Seeds and runtime}
$N_{\text{seeds}} = 5$ per control $\times$ 5 parameter sets each = 75 total runs (12.8 min).

\section{Phase II: Growth Experiment}

\subsection*{Growth via \texttt{GeneratedNetwork} (no adaptation needed)}

Because the coupled model is a standard \texttt{GeneratedNetwork}, the existing growth operators apply directly:

\begin{enumerate}
\item \textbf{Growth = adding reactions to the network.} The \texttt{NetworkGenerator.generate\_progressive()} and \texttt{generate\_progressive\_aligned()} functions add autocatalytic reactions involving the existing dynamic species ($X_1, Y_1, X_2, Y_2, E$) plus new species. Each addition extends the reaction list and species set.

\item \textbf{Growth scope: within-core only} (initial design). Added reactions involve species from a single core ($X_i, Y_i$) plus new species $Z_j$. Cross-core reactions ($X_1 + Y_2 \to \ldots$) and $E$-mediated additions are excluded initially to keep interpretation clean. This can be relaxed in a follow-up analysis.

\item \textbf{Aligned growth screening.} Candidate reactions are filtered using \texttt{passes\_oscillation\_filter()} applied to the full coupled network (not just the single-core subsystem). This tests whether the addition preserves oscillatory dynamics in the modular context.

\item \textbf{Stoichiometric rank.} $r_s$ is computed from the full stoichiometric matrix of the extended network (all reactions, all species) via \texttt{StoichiometricAnalyzer}, exactly as in Papers I--II.

\item \textbf{D$_2$ estimation.} \texttt{ActivationTracker.analyze\_network()} computes $D_2$ from the post-transient trajectory. Monotonic accumulators (waste species) are auto-excluded. \texttt{species\_to\_track} set to $\{X_1, Y_1, X_2, Y_2, E\}$ for the projected $D_2$; full-trajectory $D_2$ computed as sensitivity check.
\end{enumerate}

\subsection*{Design}

At the best parameter set from Phase I, apply $k = 0, 1, 2, 3, 4, 5$ reaction additions under two rules:
\begin{itemize}
\item \textbf{Random growth}: reactions added without screening.
\item \textbf{Aligned growth}: reactions filtered for oscillation preservation (same filter as Paper 2).
\end{itemize}

\subsection*{Sample size}

$N = 50$ trajectories per $(k, \text{rule})$ cell. Total: $6 \times 2 \times 50 = 600$ runs.

At 2 additional parameter sets (one from each pathway if both exist): $6 \times 2 \times 30 = 360$ runs for replication.

Grand total: 960 runs ($\sim$5 hours).

\subsection*{Endpoints}

Per trajectory:
\begin{itemize}
\item $D_2$ (projected onto $\{X_1, Y_1, X_2, Y_2, E\}$ and full trajectory).
\item Oscillation survival: binary (does the trajectory pass the oscillation filter?).
\item $r_s$: stoichiometric rank of the extended network.
\item $\eta = D_2 / r_s$: activation ratio (primary measure of the paper series).
\item Regime classification: fixed\_point / phase\_locked / complex.
\end{itemize}

\subsection*{Pre-registered predictions}

\begin{enumerate}
\item \textbf{$\eta$ dilution persists}: $\eta$ decreases with $k$ for both growth rules (consistent with Papers 1--2).
\item \textbf{Aligned growth preserves $D_2 > 1$}: The fraction of aligned trajectories retaining $D_2 > 1.2$ at $k = 5$ exceeds random by a factor $\ge 3$ (analogous to Paper 2's OR = 7.0 for oscillation survival).
\item \textbf{Modular coupling amplifies the alignment advantage}: The aligned-vs-random odds ratio for $D_2 > 1.2$ in the coupled system exceeds the odds ratio for oscillation survival in the single Brusselator (Paper 2). This would demonstrate that modular architecture creates a richer dynamical landscape that benefits more from topology-aware growth.
\item \textbf{$D_2$ ceiling}: Even with aligned growth, $D_2$ does not exceed $\sim$2.0 (torus ceiling for 5D system with one slow mode).
\end{enumerate}

\subsection*{Growth experiment results (COMPLETE)}

960 runs completed (primary 600, replication1 360). Replication2 ($J=7$, $\gamma=0.001$) killed after 12 hours---aligned filter intractable at strong coupling (itself an informative result: maximally rigid systems reject all perturbations).

\textbf{Dominant signal: catastrophic failure, not gradual degradation.} Random growth destroys system viability with near-certainty at $k \ge 3$:

\begin{center}
\begin{tabular}{lrrrl}
\toprule
$k$ & Random survival & Aligned survival & Fisher $p$ & OR \\
\midrule
0 & 80/80 (100\%) & 80/80 (100\%) & 1.000 & --- \\
1 & 37/80 (46\%) & 69/80 (86\%) & $<0.001$ & 0.137 \\
2 & 22/80 (28\%) & 53/80 (66\%) & $<0.001$ & 0.193 \\
3 & 12/80 (15\%) & 48/80 (60\%) & $<0.001$ & 0.118 \\
4 & 7/80 (9\%) & 53/80 (66\%) & $<0.001$ & 0.049 \\
5 & 1/80 (1\%) & 53/80 (66\%) & $<0.001$ & 0.006 \\
\bottomrule
\end{tabular}
\end{center}

\textbf{Prediction scorecard}:
\begin{enumerate}
\item $\eta$ dilution persists: \textbf{confirmed}---monotonic decrease from 0.136 to 0.079 (aligned).
\item Aligned preserves $D_2 > 1.2$ better: \textbf{nuanced}---aligned preserves \emph{viability} ($p < 0.001$), not $D_2$ among survivors (survivor bias: the few random survivors are the most robust networks).
\item Modular coupling amplifies alignment advantage: \textbf{confirmed}---OR = 0.006 at $k = 5$ (Paper 2 had OR = 7.0 for oscillation survival; Paper 3's OR = $1/0.006 = 167$ for system survival, though measured differently).
\item $D_2$ ceiling $\sim$2.0: \textbf{confirmed}---max $D_2 = 1.939$.
\end{enumerate}

\textbf{Emergent finding: growth-fragility thesis.} Unconstrained chemical elaboration almost certainly destroys the dynamical prerequisites for dimensional inflation. Only oscillation-aligned growth maintains viable dynamics, and even then, complexity ($D_2$, $\eta$) degrades monotonically. Implication for origins: chemical complexification without functional alignment is a dead end.

\section{Statistical Analysis}

\subsection*{Primary analyses}

\begin{itemize}
\item \textbf{Logistic regression}: $\text{logit}\, P(D_2 > 1.2) = \beta_0 + \beta_1 k + \beta_2 \,\mathbb{1}[\text{aligned}] + \beta_3 (k \times \text{aligned})$.
\item \textbf{Fisher exact tests}: Per $k$ level, $2 \times 2$ table (aligned vs.\ random) $\times$ ($D_2 > 1.2$ vs.\ $\le 1.2$).
\item \textbf{Bootstrap CIs}: 10{,}000 resamples for median $D_2$ at each $(k, \text{rule})$ cell.
\item \textbf{Interaction test}: Does the alignment advantage (OR) increase with modular coupling strength? Compare ORs at two coupling strengths.
\end{itemize}

\subsection*{Dimensional retention analysis}

In addition to the binary $D_2 > 1.2$ endpoint, track the \emph{retention slope}: the fraction of runs retaining $D_2 > 1.2$ as a function of $k$, separately for aligned and random growth. This is the dimensional analogue of Paper 2's oscillation survival curve.

\begin{itemize}
\item \textbf{Retention slope}: Linear fit of $P(D_2 > 1.2)$ vs.\ $k$ for each growth rule. Report slope $\pm$ SE. The key comparison: is the aligned retention slope shallower (less negative) than random?
\item \textbf{$\eta$ slope}: Linear fit of median $\eta$ vs.\ $k$. Papers 1--2 found $\approx -0.025$/reaction. Paper 3 prediction: same dilution rate persists (growth rigidity is universal), but the \emph{starting point} is higher ($D_2 > 1$ instead of $D_2 \approx 1$).
\item \textbf{Cross-paper comparison}: Tabulate the aligned-vs-random OR for three endpoints side by side: oscillation survival (Paper 2), $D_2 > 1.2$ retention (Paper 3), and $\eta > \text{threshold}$ retention (Paper 3). This makes the series progression explicit in a single table.
\end{itemize}

\subsection*{Multiple comparison correction}

Bonferroni correction across the 6 $k$-level Fisher tests ($\alpha_{\text{adj}} = 0.05/6 = 0.0083$).

\subsection*{Effect size reporting}

All results report odds ratios with 95\% CIs, not just $p$-values. Consistent with Papers 1--2 reporting conventions.

\section{Expected Outcomes}

\begin{enumerate}
\item Single oscillator with fast energy: $D_2 \approx 1$ --- energy is dynamically slaved (Pilots 1--3).
\item Single oscillator + sufficiently slow energy ($\gamma \le 0.001$): transient $D_2 \approx 1.4$ for $\sim$500--750 oscillator periods --- timescale separation enables a metastable high-dimensional phase (Sweep A2, confirming Pilot 3b's mechanism at broader parameters).
\item Two oscillators + shared slow energy (enzyme-complex): transient $D_2 > 1.2$ across a wide parameter region (Sweep A). Sharing reshapes the transient inflation map: (a) at $\gamma = 0.001$, sharing is incidental (Sweep A2); (b) at moderate $\gamma$, sharing IS causally essential---independent reservoirs produce only $D_2 \approx 0.99$, while shared reservoirs reach transient $D_2$ up to 1.67 through inter-oscillator desynchronization (confirmed by Sweep A2b). Sliding-window diagnostics confirm all high-$D_2$ phases are transient ($T_{\text{lock}} \approx 7{,}500$--$10{,}000$ time units).
\item Controls: uncoupled and diffusively-coupled oscillators without slow energy fail to produce $D_2 > 1.2$ even transiently. Independent-reservoir oscillators \emph{do} produce transient $D_2 > 1.2$ at $\gamma = 0.001$ --- confirming the intrinsic slow-energy mechanism.
\item Growth + energy coupling: $\eta$ dilutes (consistent with Papers 1--2) but aligned growth preserves transient $D_2 > 1$ and may prolong $\tau_{>1.2}$ significantly better than random growth.
\end{enumerate}

\section*{Central Thesis}

Autocatalytic oscillators are dynamically rigid: \emph{dimensionality of state space does not determine dimensionality of attractor}. A single oscillator absorbs energy perturbations without inflating $D_2$---this rigidity manifests as growth dilution (Paper I), topological invariance (Paper II), and energetic slaving (Pilot 3).

\textbf{Timescale separation and shared energetic coupling enable long-lived transient high-dimensional dynamics (``exploratory phases'') before eventual phase-locking restores rigidity.} A sufficiently slow energy variable ($1/\gamma \gg$ oscillator period) coupled to a fast autocatalytic oscillator creates genuine slow--fast dynamics that transiently inflate $D_2$ from $\sim$1 to $\sim$1.6 for hundreds of oscillator periods before the system eventually phase-locks (sliding-window diagnostics). The transient is not an artifact---Lyapunov exponents are positive during the exploratory phase, confirming genuine transient chaos.

The trilogy progression becomes:
\begin{enumerate}
\item \textbf{Structural rigidity} (Paper I): Adding reactions dilutes $\eta$.
\item \textbf{Topological rigidity} (Paper II): Aligned growth preserves oscillation but not $D_2$.
\item \textbf{Transient escape from rigidity} (Paper III): A slow energetic degree of freedom enables a long-lived metastable high-dimensional phase before rigidity is restored.
\end{enumerate}

The mechanism operates at two levels, now confirmed by independent-reservoir controls (Sweeps A2 + A2b) and characterized temporally by sliding-window diagnostics:
\begin{itemize}
\item \textbf{Slow energy alone} transiently inflates $D_2$ to $\sim$1.4 (each 3D subsystem independently, $\gamma = 0.001$; Sweep A2). Transient lasts $\sim$500--750 oscillator periods before phase-locking.
\item \textbf{Shared slow energy} produces qualitatively richer transient inflation ($D_2$ up to 1.67) at moderate $\gamma = 0.002$--$0.003$ through inter-oscillator desynchronization. Independent subsystems produce only $D_2 \approx 1.0$ at these $\gamma$ values (Sweep A2b). \textbf{Sharing is causally essential here}, though the complex phase is still transient ($\sim$250--500 oscillator periods).
\item Prebiotic chemical systems do not require infinite-time chaos. What matters is finite-time exploration: a metastable high-dimensional phase increases the ``search'' over chemical state space before the system settles. Selection could then act on networks that prolong $\tau_{>1.2}$---the duration of the exploratory phase.
\end{itemize}

\section*{Appendix: Phase Summary and Timeline}

\begin{center}
\begin{tabular}{llrl}
\toprule
Phase & Description & Est.\ runs & Est.\ wall time \\
\midrule
Pilot 5 & Pure mass-action validation (\textbf{failed}) & 75 & $\sim$20 min \\
Pilot 5b & Enzyme-complex validation (\textbf{validated}) & 117 & $\sim$39 min \\
0.5 & Ridge mapping (subsumed by Sweep A) & --- & --- \\
I-A & Fine grid Sweep A (\textbf{done}: 22/100 complex) & 500 & 126 min \\
I-A2 & Independent-reservoir control at $\gamma = 0.001$ (\textbf{done}: D$_2$ persists) & 60 & 32 min \\
I-A2b & Independent-reservoir control at moderate $\gamma$ (\textbf{done}: D$_2$ disappears) & 40 & 21 min \\
Diag & Diagnostics (\textbf{done}: transient chaos, $T_{\text{lock}} \approx 7.5$--10k) & 8 & 38 min \\
I-B/C & Sweeps B \& C (\textbf{deferred}: gate sensitivity, not needed for thesis) & 75 & $\sim$25 min \\
I-D & Control experiments (\textbf{done}: all 4 controls passed) & 75 & 12.8 min \\
II & Growth experiment (\textbf{done}: 960 runs, growth-fragility thesis) & 960 & $\sim$16 h \\
--- & Statistical analysis \& writing & --- & $\sim$1 week \\
\bottomrule
\end{tabular}
\end{center}

Total compute: $\sim$15 hours (parallelized). Total calendar: $\sim$2--3 weeks.

\end{document}
